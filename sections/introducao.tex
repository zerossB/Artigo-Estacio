\paragraph{}
Nos dias atuais é quase que impossível não utilizarmos um \textit{software}, seja ele o \textit{driver} que mostra o vídeo no \textit{display} do seu smartfone,
até a aplicação web que você acessa para fazer compras online, tudo tem um programa, ou um script por trás que possa fazer tudo funcionar,
porém como você tem a permissão de utiliza-lo?

\paragraph{}
Sabemos que para tudo na vida tem leis e regras, no mundo da computação não é diferente, há algo chamado licenças,
as licenças de \textit{software} livre são documentos através dos quais os detentores dos direitos sobre um programa de computador
autorizam usos de seu trabalho que, de outra forma, estariam protegidos pelas leis vigentes no local \cite{sabino2009licenccas}.
No caso do \textit{software} Livre temos os 4 direitos básicos que o \textit{software} pode ser usado, copiado, estudado, modificado e redistribuído
sem restrição. A forma usual de um \textit{software} ser distribuído livremente é sendo acompanhado por uma licença de \textit{software} livre
(como a GPL ou a BSD), e com a disponibilização do seu código-fonte \cite{campos2006software}.

\paragraph{}
Se formos pesquisar sobre licenças de \textit{software} vamos nos deparar com muitos tipos para todos os casos, desde a mais permissiva
até a mais liberativa. Más qual o impacto de cada uma dentro de uma comunidade de desenvolvedores? Para isso, vamos utilizar
o maior hub de código do mundo atualmente, GitHub, lá podemos encontrar mais de 5 Milhões de projetos hospedados
(sendo públicos, ou privados). Qual o impacto de uma licença dentro de um projeto de sucesso? Esse trabalho serve para
estudar isso, definir o qual o impacto que uma licença tem em cima de um \textit{software}.